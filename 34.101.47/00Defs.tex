\chapter{Обозначения}\label{DEFS}

\section{Список обозначений}

{\tabcolsep 0pt
\begin{longtable}{lrp{13.5cm}}
$\{0,1\}^n$  & \hspace{2mm} &
множество всех слов длины $n$ в алфавите~$\{0,1\}$;
\\[4pt]
$\{0,1\}^*$  &&
множество всех слов конечной длины в алфавите~$\{0,1\}$
(включая пустое слово длины $0$);
\\[4pt]
%
$\{0,1\}^{n*}$  &&
множество всех слов из~$\{0,1\}^*$,
длина которых кратна~$n$;
\\[4pt]
%
$|u|$      &&
длина слова $u\in\{0,1\}^*$;
\\[4pt]
%
$\alpha^n$  &&
слово длины $n$ из одинаковых символов $\alpha\in\{0,1\}$;
\\[4pt]
%
$u\parallel v$  &&
конкатенация
$u_1 u_2\ldots u_n v_1 v_2\ldots v_m$
слов
$u=u_1 u_2\ldots u_n$ и
$v=v_1 v_2\ldots v_m$;
\\[4pt]
%
$\texttt{01234\ldots}_{16}$ && 
представление $u\in\{0,1\}^{4*}$ шестнадцатеричным словом,
при котором последовательным четырем символам~$u$ соответствует
один шестнадцатеричный символ
(например, $10100010=\texttt{A2}_{16}$);
\\[4pt]
%
$\texttt{\textquotedbl 01234\textquotedbl}$ && 
строка графических символов базовой таблицы KOИ-7, 
представляющая слово $u\in\{0,1\}^{8*}$ так,
что каждому символу строки соответствует один октет~$u$
(например, $\texttt{\textquotedbl A2\textquotedbl}=0100000100110010$);
\\[4pt]
%
$x\bmod m$             &&
для целого~$x$ и натурального $m$ остаток от деления $x$ на $m$,
т.~е. число $r\in\{0,1,\ldots,m-1\}$ такое, что $m$ делит $x-r$;
\\[4pt]
%
$u\oplus v$             &&
для~$u=u_1 u_2\ldots u_n\in\{0,1\}^n$ 
и~$v=v_1 v_2\ldots v_n\in\{0,1\}^n$
слово~$w=w_1 w_2\ldots w_n\in\{0,1\}^n$
из символов~$w_i=(u_i+v_i)\bmod{2}$;
\\[4pt]
%
$\bar u$                &&
а)~для~$u=u_1 u_2\ldots u_8\in\{0,1\}^8$
число $2^7 u_1+2^6 u_2+\ldots+u_8$ и\\[2pt]
%
                        &&
б)~для~$u=u_1\parallel u_2\parallel\ldots\parallel u_n$, $u_i\in\{0,1\}^8$,
число~$\bar u_1+2^8\bar u_2+\ldots+2^{8(n-1)}\bar u_n$;
\\[4pt]
%
$\langle U\rangle_{8n}$ &&
для целого~$U$ 
слово $u\in\{0,1\}^{8n}$ такое, что $\bar u=U\bmod 2^{8n}$;
\\[4pt]
%
$u\boxplus v$           &&
для~$u,v\in\{0,1\}^{8n}$ слово $\langle\bar u+\bar v\rangle_{8n}$;
\\[4pt]
%
$c\leftarrow u$         &&
присвоение переменной $c$ значения $u$;
\\[4pt]
%
$\HMAC[h]$         &&
определенный в~\ref{HMAC}
алгоритм выработки имитовставки на основе 
допустимой функции хэширования~$h$.
\end{longtable}
} % tabcolsep
\setcounter{table}{0}

\section{Пояснения к обозначениям}

\subsection{Слова}

Двоичные слова представляют собой последовательности символов из 
алфавита~$\{0,1\}$. Символы нумеруются слева направо от единицы.
%
В настоящем подразделе в качестве примера рассматривается слово
$$
w=1011 0001 1001 0100 1011 1010 1100 1000.
$$
В этом слове первый символ~--- $1$, 
второй~--- $0$, \ldots, последний~--- $0$.

Слова разбиваются на тетрады из четверок последовательных двоичных символов.
%
Тетрады кодируются шестнадцатеричными символами по следующим правилам
(см. таблицу~\ref{Table.Hex}):

\begin{table}[h]
\caption{Шестнадцатеричные символы}\label{Table.Hex}
\begin{tabular}{|c|c||c|c||c|c||c|c|}
\hline
тетрада & символ & тетрада & символ & тетрада & символ & тетрада & символ\\
\hline
\hline
0000 & $\texttt{0}_{16}$ & 0001 & $\texttt{1}_{16}$ & 
0010 & $\texttt{2}_{16}$ & 0011 & $\texttt{3}_{16}$\\
0100 & $\texttt{4}_{16}$ & 0101 & $\texttt{5}_{16}$ & 
0110 & $\texttt{6}_{16}$ & 0111 & $\texttt{7}_{16}$\\ 
1000 & $\texttt{8}_{16}$ & 1001 & $\texttt{9}_{16}$ & 
1010 & $\texttt{A}_{16}$ & 1011 & $\texttt{B}_{16}$\\ 
1100 & $\texttt{C}_{16}$ & 1101 & $\texttt{D}_{16}$ & 
1110 & $\texttt{E}_{16}$ & 1111 & $\texttt{F}_{16}$\\ 
\hline
\end{tabular}
\end{table}

Пары последовательных тетрад образуют октеты.
Последовательные октеты слова~$w$ имеют вид:
$$
1011 0001=\texttt{B1}_{16},\ 
1001 0100=\texttt{94}_{16},\ 
1011 1010=\texttt{BA}_{16},\  
1100 1000=\texttt{C8}_{16}.
$$

Некоторые октеты могут кодироваться графическими символами базовой 
таблицы КОИ-7, определенной в ГОСТ~27463. 
Правила кодирования заданы в таблице~\ref{Table.ASCII}.
%
В таблице используется шестнадцатеричное представление 
слов~$u\in\{0,1\}^8$. Если $u=\texttt{IJ}_{16}$, то символ, 
представляющий~$u$, находится на пересечении строки $\texttt{I}$ и 
столбца~$\texttt{J}$. 
%
Октет $\texttt{20}_{16}$ кодируется пробелом,
октет $\texttt{7F}_{16}$ не кодируется графическим символом.

\begin{table}[bht]
\caption{Графические символы базовой таблицы КОИ-7}\label{Table.ASCII}
{\small\tt
{\tabcolsep=8.5pt
\begin{tabular}{|l|rrrrrrrrrrrrrrrr|}
\hline
 &  0&  1&  2&  3&  4&  5&  6&  7&  8&  9&  A&  B&  C&  D&  E&  F\\
\hline
2&   &  !&  "&  \#& \$& \%& \&&  '&  (&  )&  *&  +&  ,&  -&  .&  /\\
3&  0&  1&  2&  3&  4&  5&  6&  7&  8&  9&  :&  ;&  <&  =&  >&  ?\\
4&  @&  A&  B&  C&  D&  E&  F&  G&  H&  I&  J&  K&  L&  M&  N&  O\\
5&  P&  Q&  R&  S&  T&  U&  V&  W&  X&  Y&  Z&  [& \textbackslash &]&\textasciicircum&\_\\
6&  `&  a&  b&  c&  d&  e&  f&  g&  h&  i&  j&  k&  l&  m&  n&  o\\
7&  p&  q&  r&  s&  t&  u&  v&  w&  x&  y&  z& \{&  |& \}&  \textasciitilde&   \\
\hline
\end{tabular}
} % tabcolsep
} % tt
\end{table}

\subsection{Слова как числа}

Октету $u=u_1 u_2\ldots u_8$ ставится в соответствие байт~--- 
число $\bar{u}=2^7u_1+2^6 u_2+\ldots + u_8$. 
Например, октетам $w$ соответствуют байты
$$
177=2^7+2^5+2^4+1,\ 
148=2^7+2^4+2^2,\ 
186=2^7+2^5+2^4+2^3+2^1,\ 
200=2^7+2^6+2^3.
$$

Число ставится в соответствие не только октетам, но и любому другому
двоичному слову, длина которого кратна~$8$. 
%
При этом используется распространенное для многих современных 
процессоров соглашение <<от младших к старшим>> (little-endian):
считается, что первый байт является младшим, последний~--- старшим.
Например, слову $w$ соответствует число
$$
\bar{w}=177+2^{8}\cdot 148+2^{16}\cdot 186+2^{24}\cdot 200 = 3367670961.
$$


