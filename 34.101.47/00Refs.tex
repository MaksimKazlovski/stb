\chapter{Нормативные ссылки}

В настоящем cтандарте использованы ссылки на следующие 
технические нормативные правовые акты в области 
технического нормирования и стандартизации (далее~--- ТНПА):

СТБ 1176.1-99 Информационная технология. Защита информации.
Функция хэширования

СТБ 34.101.19-2011 Информационные технологии. 
Форматы сертификатов и списков отозванных сертификатов 
инфраструктуры открытых ключей

СТБ 34.101.31-2011 Информационные технологии и безопасность. 
Криптографические алгоритмы шифрования и контроля целостности

СТБ 34.101.65-2014 Информационные технологии и безопасность. 
Протокол защиты транспортного уровня (TLS)

ГОСТ 34.973-91 (ИСО 8824-87) Информационная технология. Взаимосвязь
открытых систем. Спецификация абстрактно-синтаксической нотации
версии 1 (АСН.1)

ГОСТ 27463-87 Система обработки информации. 7-битные кодированные наборы 

\begin{list}{}
{\setlength{\labelwidth}{0pt}\setlength{\topsep}{0pt}
\setlength{\partopsep}{0pt}\setlength{\parskip}{0pt}
\setlength{\itemindent}{0pt}\setlength{\leftmargin}{\parindent}
\setlength{\labelsep}{0pt}}
\item[]
{\small Примечание~---~При пользовании настоящим стандартом
целесообразно проверить действие ТНПА по каталогу, 
составленному по состоянию на 1 января текущего
года, и по соответствующим информационным указателям, опубликованным
в текущем году.\\
Если ссылочные ТНПА заменены (изменены), то при
пользовании настоящим стандартом следует руководствоваться
замененными (измененными) ТНПА. Если ссылочные ТНПА отменены без
замены, то положение, в котором дана ссылка на них, применяется в
части, не затрагивающей эту ссылку.
}
\end{list}

