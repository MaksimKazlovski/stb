\clearpage
\begin{thebibliography}{999}
\bibitem{RFC-HMAC}
Krawchuk~H., Bellare~M., Canetti~R.
HMAC: Keyed-Hashing for Message Authentication\\
{\small Request for Comments: 2104, 1997.}

\bibitem{HOTP}
M'Raihi D., 
Bellare M.,
Hoornaert F.,
Naccache D.,
Ranen O.
HOTP: An HMAC-Based One-Time Password Algorithm\\
{\small Request for Comments: 4226, 2005.}

\bibitem{TOTP}
M'Raihi D., 
Machani S.,
Pei M.,
Rudell J.
TOTP: Time-Based One-Time Password Algorithm\\
{\small Request for Comments: 6238, 2011.}

\bibitem{OCRA}
M'Raihi D., 
Rudell J.,
Bajaj S.,
Machani S.,
Naccache D.
OCRA: OATH Challenge-Response Algorithm\\
{\small Request for Comments: 6287, 2011.}

\bibitem{UTF8}
ISO/IEC 10646:2012
Information technology~-- Universal Coded Character Set (UCS)\\
{\small International Organization for Standardization, 2012}\\
{\small (Информационные технологии. 
Универсальный набор кодированных символов (UCS))}

\bibitem{RD-PRNG}
РД РБ 07040.1206-2003. 
Руководящий документ Республики Беларусь. 
Автоматизированная система межбанковских расчетов. 
Процедура выработки псевдослучайных данных с использованием 
секретного параметра\\
{\small Мн.: Национальный Банк Республики Беларусь, 2003.}

\label{LastBib}
\end{thebibliography}

