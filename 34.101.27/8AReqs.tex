\chapter{Гарантийные требования безопасности}\label{AReqs}

\section{Требования по проектированию и разработке (ПР)}

\req{ПР}{1, 2}\label{ProgSpec}
Должно быть дано описание программ~\TOE и установлено соответствие 
между функциональной спецификацией и средствами безопасности,
реализованными в программах.

\req{ПР}{1, 2}\label{HLD}
В описании программ~\useref{ProgSpec}
должны быть определены все внешние интерфейсы~\TOE.

\req{ПР}{2}\label{LLD}
В описании программ~\useref{ProgSpec}
должны быть определены внутренние компоненты~\TOE и их интерфейсы.

\req{ПР}{1, 2}\label{Tools}
В описании программ~\useref{ProgSpec}
должны быть определены все используемые средства разработки
и сборки программ. Должны быть перечислены все конфигурационные файлы,
отвечающие за настройку средств разработки и сборки. 
Конфигурационные файлы должны быть включены в список 
элементов конфигурации~\forref{CMList}.

\req{ПР}{1, 2}\label{Comments}
Исходные тексты программ должны быть снабжены комментариями,
устанавливающими соответствие с описанием программ~\forref{ProgSpec}.

\req{ПР}{1, 2}\label{Language}
Программы должны быть написаны на высокоуровневых языках программирования.
Вставки на низкоуровневых языках (языках ассемблера)
допускаются в случаях, критичных для производительности, 
а также тогда, когда высокоуровневые языки применить нельзя.


\section{Требования по поддержке жизненного цикла (ЖЦ)}

\req{ЖЦ}{1, 2}\label{CMSystem}
Должна быть определена и реализована система управления конфигурацией для \TOE.
Система должна обеспечивать:
\begin{itemize}
\item[--]
контроль доступа разработчиков к элементам конфигурации;
\item[--]
контроль версий элементов конфигурации;
\item[--]
отслеживание изменений элементов конфигурации;
\item[--]
сборку программ по исходным текстам~\useref{Tools}.
\end{itemize}

\req{ЖЦ}{1, 2}\label{CMList}
В перечень элементов конфигурации должны быть включены:
\begin{itemize}
\item[--]
функциональная спецификация;
\item[--]
программы;
\item[--]
описание программ~\useref{ProgSpec};
\item[--]
исходные тексты программ;
\item[--]
документация по управлению конфигурацией~\useref{CMSystem};
\item[--]
документация по поставке~\TOE потребителю~\useref{Delivery};
\item[--]
документация по устранению недостатков~\useref{FlawRemediation}
(только для класса 2);
\item[--]
руководства~\useref{AdminGuide}, \useref{UserGuide}.
\end{itemize}

\req{ЖЦ}{1, 2}\label{CMVersion}
Каждая версия каждого элемента конфигурации 
должна быть снабжена уникальным идентификатором. 

\req{ЖЦ}{1, 2}\label{Delivery}
Должна быть определена и реализована система поставки~\TOE потребителю.  

\req{ЖЦ}{2}\label{Authenticode}
Должны быть предусмотрены средства контроля целостности и подлинности 
инсталляционных программ после их доставки потребителю. 

\req{ЖЦ}{2}\label{FlawRemediation}
Должна быть определена и реализована система устранения недостатков 
в программах и документации~\TOE.
Система должна обеспечивать:
\begin{itemize}
\item[--]
регистрацию недостатков;
\item[--]
определение порядка выявления причин недостатков
и исправления недостатков;
\item[--]
отслеживание статуса недостатков 
(подтвержден, исправляется, исправлен и др.);
\item[--]
описание способа устранения недостатков;
\item[--]
порядок извещения потребителей об устранении недостатков.
\end{itemize}

\section{Требования к руководствам (РД)}

\req{РД}{1, 2}\label{AdminGuide}
Должно быть разработано руководство администратора.
Руководство должно описывать:
\begin{itemize}
\item[--]
обязанности администратора по настройке среды~\useref{ENVInstall}, 
\useref{ENVObjects}, \useref{ENVSession};
\item[--]
инструкции по установке~\TOE~\useref{ENVInstall};
\item[--]
доступные администратору сервисы~\useref{DAC};
\item[--]
обязанности администратора по настройке средств безопасности~\TOE;
\item[--]
связанные с безопасностью предположения 
относительно поведения операторов.
\end{itemize}

\req{РД}{1, 2}\label{UserGuide}
Для каждой роли~\useref{Roles}, отличной от роли <<Администраторы>>, 
должно быть разработано руководство ее операторов.
Руководство должно определять:
\begin{itemize}
\item[--]
доступные оператору сервисы~\useref{DAC};
\item[--]
обязанности оператора по обеспечению безопасности~\TOE.
\end{itemize}

\req{РД}{2}\label{Misuse}
Руководства~\useref{AdminGuide}, \useref{UserGuide} 
должны описывать типичные ошибки операторов,
которые могут привести к снижению безопасности~\TOE.
Руководства должны давать рекомендации операторам по избежанию
ошибок.

\section{Требования по программе испытаний (ПИ)}

\req{ПИ}{1, 2}\label{TestProgram}
Должна быть разработана программа испытаний~\TOE разработчиком.
Программа должна определять:
\begin{itemize}
\item[--]
планы тестирования;
\item[--]
содержание тестов;
\item[--]
ожидаемые результаты выполнения тестов;
\item[--]
фактические результаты выполнения тестов.
\end{itemize}

\req{ПИ}{1, 2}\label{TestCoverage}
Тесты программы испытаний~\useref{TestProgram} должны 
покрывать все функциональные возможности~\TOE, 
определенные в функциональной спецификации.

\req{ПИ}{2}\label{TestDeep}
Тесты программы испытаний~\useref{TestProgram} должны 
покрывать функциональные возможности всех компонентов~\TOE, 
определенных в описании программ~\useref{ProgSpec}.


