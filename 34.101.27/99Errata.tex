\clearpage
\chapter*{\mbox{}\hfill Поправка к официальной редакции\hfill\mbox{}}

\mbox{}

\begin{center}
\begin{tabular}{|p{3.5cm}|p{6cm}|p{6cm}|}
\hline
В каком месте & Напечатано & Должно быть\\
\hline
\hline
Приложение~\ref{EXAMPLE},\par
подраздел~\ref{EXAMPLE.Crypto},\par
3-й~абзац с конца &
Алгоритмы~\ref{Alg.DataWrap} и \ref{Alg.Sign} относятся к классу 
симметричных. &
Алгоритмы~\ref{Alg.DataWrap} и \ref{Alg.PRNG} относятся к классу 
симметричных.
\\
\hline
Приложение~\ref{EXAMPLE},\par
подраздел~\ref{EXAMPLE.RNG},\par
3-й~абзац с конца &
Определяются серии (последовательности одинаковых символов) различных длин. 
Пусть $S_i$~--- количество серий длины $i=1,2,\ldots$, и $S_{6+}=S_6+S_7+\ldots$.
Тест пройден, если 
$S_1\in[2315,2685]$,
$S_2\in[1114,1386]$,
$S_3\in[527,723]$,
$S_4\in[103,209]$,
$S_5,S_{6+}\in[103,209]$.
&
Определяются серии (максимальные последовательности повторяющихся соседних 
битов) различных длин. 
Тест пройден, если и для серий из нулей, и для серий из единиц выполняется: 
$S_1\in[2315,2685]$,
$S_2\in[1114,1386]$,
$S_3\in[527,723]$,
$S_4\in[240,384]$,
$S_5,S_{6+}\in[103,209]$.
Здесь~$S_i$~--- количество серий длины $i=1,2,\ldots$, 
$S_{6+}=S_6+S_7+\ldots$.

\\
\hline
\end{tabular}
\end{center}